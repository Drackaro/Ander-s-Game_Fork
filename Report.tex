%! Author = Drackaro
%! Date = 4/16/2024

% Preamble
\documentclass[11pt]{article}

%Packages
\usepackage{amsmath}
\usepackage{graphicx}
\usepackage{amssymb}
\usepackage{longtable}

\author{
    Javier Rodríguez Sanchez C-411 \\ 
    Alejandro Camacho Pérez C-412 \\ 
    Luis Alejandro Rodríguez Otero C-411
    }
\title{Proyecto Final de IA y Simulación \\ El Juego de Ander}

% Document
\begin{document}

    \maketitle
    \newpage

    \tableofcontents
    \newpage

    \section{Introducción}
        \subsection{Estrategias para guerras medievales: El juego de Ander}
        En un contexto medieval, donde hay n reinos en guerra, se busca saber 
        cuál es la mejor estrategia a seguir por los reyes para salir victoriosos. 
        En este contexto, para evitar conflictos históricos o culturales, lo 
        desarrollaremos en una tierra lejana llamada Ander, donde existen n reinos 
        anónimos. \vspace{5mm}

        \noindent Un reino consta de un rey, el pueblo al que gobierna, su castillo/fortaleza 
        y un ejército. Todos los reinos inician en guerra por conveniencia. Un reino 
        puede mejorar sus murallas o aumentar su ejército y atacar o aliarse a otros reinos. 
        El objetivo de cada uno es salir airoso de la pelea, es decir, ser el único que sobreviva.

    \newpage

    \section{Modelo utilizado}
        \subsection{Resumen del Modelo}
        El modelo usado para representar el sistema consiste en un accionar por turnos 
        de cada reino. La población estará clasificada según que tan grande es (dado por 
        niveles del 0 al 10) y su castillo/muralla tendrá un índice de fortaleza. El 
        ejército estará dividido en tropas cada una con un tamaño. La población de un reino 
        asumimos que está en constante cambio, y todos los turnos variara en tamaño, con mayor 
        probabilidad de permanecer cerca del valor anterior. Durante un turno los reyes pueden 
        ordenar a su población a reforzar sus murallas o a reclutar tropas, según permita el 
        tamaño del pueblo; y ordenar a sus tropas a atacar otros reinos. Para destruir un reino 
        debes matar a su rey. Para esto primero hay que destruir su ejército, luego sus murallas, 
        luego su pueblo y finalmente al rey. Si dos tropas pelean, la más fuerte gana, pero 
        pierde en fuerza igual al nivel de la tropa destruida. Si una tropa ataca una muralla, 
        la muralla resiste, pero pierde resistencia igual a la mitad del nivel de la tropa que la 
        asedia. Si un pueblo es atacado por una tropa, este quedará diezmado en una cantidad igual 
        al nivel de la tropa. Una simulación termina cuando solo queda un rey en pié o cuando esta
        alcanza un número de rondas predefinido, en este caso el ganador es el reino con mejor puntaje,
        definiendo puntaje como la suma entre los tamaños de todas sus tropas y su muralla. \vspace{5mm}

        \noindent Los reyes consistirán en agentes, los cuales contienen conocimiento básico de como manejar 
        a su reino y el estado actual de este y de los reinos enemigos en todo momento. Ademas, 
        cada rey tiene niveles de relación con el resto de los reinos, los cuales inician siendo 
        bajos, pero dichas relaciones pueden mejorar. Un rey solo conoce la relación que tienen 
        otros con él y las relaciones no son simétricas. Entre las cosas que afectan una relación 
        están: entrar en alianza(aumento grande), que ataquen a un enemigo(aumento mínimo), que 
        ataquen a un aliado(disminución mínima) recibir un ataque a una tropa(disminución estándar), 
        recibir un ataque a la muralla(disminución alta), recibir un ataque al pueblo(disminución enorme), 
        ser traicionado(disminución maxima). Dos reinos pueden entrar en una alianza, la cual queda 
        como un pacto de no agresión por 3 turnos. Una traición ocurre cuando un aliado ataca a otro. \vspace{5mm}

        \noindent Las decisiones sobre la alianza y que acciones serán tomadas en un turno, conforman la 
        estrategia que sigue un rey. El objetivo de nuestro modelo es encontrar la mejor estrategia.

        \subsection{Ventajas}
        Modelo simple y escalable a todos los niveles. Permite dar gran diversidad de estrategias a probar 
        en un gran número de condiciones iniciales.

        \subsection{Desventajas y Limitaciones}
        Las decisiones de los reinos no se realizan en tiempo real, lo cual afecta el realismo del modelo 
        y no toma en cuenta la velocidad de la toma de decisiones, factor crucial en una guerra. La ausencia 
        de mapa y de recursos también hace que no se tomen en cuenta costos de ataque a un reino o fortalecimiento 
        de las murallas. El conocimiento de los estados de los reinos ajenos no es realista, ya que en el sistema 
        el conocimiento de los reyes suele ser limitado. Una tropa numéricamente superior no siempre es la que 
        sale victoriosa de un combate, quitándole lugar a la incertidumbre. El carácter centralizado de los reinos 
        también es una suavización del problema, haciendo que todas las órdenes de este se ejecuten, y no hayan 
        controversias internas.

        \subsection{Aspectos Mejorables}
        Los siguientes aspectos no fueron incluidos en el modelo por falta de tiempo, pero son perfectamente 
        integrables:

        \begin{itemize}

            \item La adición de capacidades especiales: cada reino tendría una capacidad extra y distintiva 
            que le permitiera distinguirse del resto de los reinos (se pudiera ver como cierta tecnología que 
            contengan, una especie de arte marcial, un don divino o una raza), que le permita extender las 
            reglas establecidas y otorgarles cierta ventaja.

            \item La comunicación entre aliados: en una alianza, un rey puede recomendarle a un aliado suyo 
            que acciones realizar en su turno. De esta manera algunas estrategias podrían basar sus jugadas
            en las recomendaciones de sus aliados

        \end{itemize}

    \newpage

    \section{Implementación}
    El código de la aplicación que permite la simulación, y que será relatado a continuación, se encuentra 
    en la carpeta \textbf{src}.

        \subsection{Modelo de la Simulación}
        Todas las descripciones y la lógica de un reino se encuentran en la clase \textbf{Kingdom} en el archivo 
        \textbf{Simulation\_Model/Reigns.py}. De aquí vale la pena destacar el método 'actions', el cual devuelve una 
        lista con todas las órdenes inmediatas que puede dar un rey, y el método 'act', que dado una acción 
        la ejecuta. La clase está diseñada de tal manera que pueda ser extensible para agregar nuevas 
        funcionalidades al modelo en caso de ser necesario.

        \subsection{Agente}
        Los agentes del modelo serían los reyes, y están representados en la clase \textbf{Agent} del archivo 
        \textbf{Agent/Agent.py}. Estos contienen como propiedad una base de conocimientos simple, representada 
        en \textbf{Agent/Knowledge\_Base.py}.

        \subsection{Base de Conocimiento de los agentes}
        Esta cuenta con dos métodos fundamentales: 'Learn' y 'Think'. El primero permite al agente guardar 
        toda aquella información que sea útil para este, como la estrategia que seguirán, el número de 
        reinos y la información de estos, el estado actual de la geopolítica y las acciones realizadas 
        por otros reinos. El segundo permite a un agente tomar sus decisiones dado su conocimiento, como 
        las acciones que realizara durante su turno, sus consecuencias inmediatas y con que reinos aliarse.

        \subsection{Toma de decisiones}
        Las decisiones de un rey consistirán en una búsqueda informada de cuál sería el mejor camino para 
        su reino, donde buscaría entre todos los posibles finales cuáles son los que más le conviene seguir, 
        de esta forma, este elige cuáles serían las decisiones óptimas en función de su estrategia.

        \subsection{Estrategia de un agente}
        Un agente tiene una estrategia, y nuestro objetivo es encontrar la más conveniente. Todas estas 
        estrategias se encuentran en la carpeta \textbf{Strategies/Implementations} y heredan de la clase 
        \textbf{Strategy} en \textbf{Strategies/Strategy.py}. Para implementar nuevas estrategias solo deben
        crearse más clases que hereden de \textbf{Strategy} y añadirlas a esta carpeta. Cada una representa
        una forma distinta de tomar decisiones para un agente en una simulación.

        \subsection{Simulador}
        El proceso de simular consta de dos componentes principales:

        \begin{itemize}

            \item Una interfaz que sirve de intermediario entre el usuario y la simulación.
            
            \item Un simulador que permite correr varios juegos y obtener estadísticas de estos.
            
        \end{itemize}

        \subsection{Interfaz}
        La interfaz de la simulación está creada para que pueda ser extensible a cualquier tipo de 
        visualización que ofrezca python. Esto se logra con una clase abstracta \textbf{SimulationInterface} 
        la cuál contiene tanto los métodos necesarios para poder interactuar con el usuario, cómo los 
        métodos para poder correr una simulación. \vspace{5mm}

        \noindent En este trabajo, la interfaz está a cargo de la clase \textbf{SimulationInterfaceConsole}, la cuál permite 
        correr la simulación en la consola de python. Esta clase se encarga de mostrar los resultados de la 
        simulación, así como de pedir los parámetros necesarios para correr una simulación. Ambas clases se encuentran
        en \textbf{Simulation\_gestor.simulation\_interface.py}

        \subsection{Llm}
        Para un mejor análisis de los resultados, utilizamos \textbf{LLM} para resumir el historial del juego y 
        crear una historia más amena y legible en vista a la interpretación de las decisiones tomadas por 
        los agentes, y entender qué decidieron hacer y en qué contexto. Dando la facilidad de tener una 
        idea del desarrollo de la simulación.\vspace{5mm}

        \noindent Con la finalidad de lograr esto, se crea un proceso que va a estar corriendo en paralelo, el cual 
        es el encargado tanto de ir gestionando los logs que generó un juego, como de comunicarse con el 
        proveedor de LLM para poder ir generando la historia del juego.\vspace{5mm}

        \noindent Para lograr lo más cercano posible a lo deseado, a la hora de hacer la petición al servidor LLM, 
        como parámetro del sistema se pasa un resumen de lo que ha acontecido de historia hasta el momento, 
        así como cosas que son invariantes en la historia que se está generando, y que de perderse, puede 
        carecer de sentido lo generado. Luego, se pasa como petición del cliente un prompt el cual es creado 
        a partir de la información extraida del log del juego que se desea decorar.\vspace{5mm}

        \noindent Para establecer la comunicación entre el proceso del simulador y el proceso creador de la historia, 
        ambos procesos comparten una misma cola de logs, que utiliza la clase \textbf{Queue}, de la biblioteca 
        \textbf{multiprocessing}, ya que esta garantiza seguridad en la comunicación entre procesos. Además, el 
        proceso de la historia recibe del proceso del simulador, un delegado que le va a permitir la gestión 
        del log en la interfaz de la historia.

    \newpage

    \section{Análisis de las simulaciones realizadas}
        \subsection{Estrategias implementadas}
        Volviendo a la pregunta inicial: ¿cuál es la mejor estrategia?. Para saber esto, buscamos resolverlo 
        inicialmente para el caso n=2 y n=3 (n es la cantidad de jugadores) en igualdad de condiciones iniciales.
        Para cada simulación se estableció un límite de 50 rondas y se anotó el ganador de cada una. Las estrategias
        implementadas actualmente son las siguientes:

        \begin{itemize}
            \item \textbf{Allies Strategy:} Toma Decisiones de acuerdo al nivel de relación que tiene con el 
            resto de reinos. De esta forma, tiene una alta probabilidad de atacar a los reinos con los que se 
            lleva mal y propone alianzas a aquellos con los que tiene buenas relaciones.

            \item \textbf{Always Attack Strategy:} Ataca a la mayor cantidad posible de objetivos, entiéndase 
            tropas o murallas enemigas. No propone alianzas y solo acepta proposiciones de esta a reinos con los 
            que tiene buenas relaciones.

            \item \textbf{Attack Weak Strategy:} Ataca al reino con el ejército más débil en ese momento. Aunque 
            por momentos su comportamiento es equivalente al de \textbf{Always Attack}, la principal diferencia 
            entre estas dos estrategias está en la manera que elijen aliados. La anterior era bastante básica pero 
            esta solo se alía con el reino del ejército más fuerte del momento. Por lo que, resumiendo, esta es una 
            estrategia que ataca al más débil y se alía con el más fuerte.

            \item \textbf{Current Situation Strategy:} Crea una tabla en donde ubica de forma ordenada a los reinos 
            respecto a su poder (teniendo en cuenta tropas y murallas) y actúa dependiendo de su puesto en dicha tabla. 
            Por ejemplo si es el primero de la tabla, es decir, si es el reino más poderoso de ese momento, tiene un 70\% 
            de posibilidad de solo defenderse y un 30\% de posibilidad de atacar al que esté en 2do puesto (Este 30\% disminuye 
            si el objetivo tiene buenas relaciones con él o son aliados), si por el contrario es el último de la tabla entonces 
            intentará atacar al penúltimo para mejorar su posición, y si se encuentra en medio de la tabla puede atacar al que 
            está por delante o al que está por detrás. Cada vez que elije atacar a alguien primero comprueba su relación y estado 
            de alianza hacia este, lo cual puede hacerlo no atacar y optar por un final defensivo. Respecto a las alianzas, 
            este solo propone alianzas cuando es el reino más poderoso, y solo acepta proposiciones de reinos que no están en 
            una posición adyacente a la suya en la tabla, ya que estos son usualmente sus objetivos a atacar.

            \item \textbf{Defend Strategy:} Sube de nivel su muralla y su ejército en cada turno y opta por no atacar casi nunca. 
            Tiene un 20\% de posibilidades de atacar un reino con el que tenga muy malas relaciones, lo cual es la única situación 
            en que esta estrategia es ofensiva. No propone alianzas y acepta solicitudes solo de reinos con los que tiene buenas 
            relaciones.

            \item \textbf{Do Nothing Strategy:} Pasa cada turno sin hacer ninguna acción, no propone, ni acepta alianzas. Es normal 
            preguntarse qué hace esta estrategia aquí, pero fue bastante útil a la hora de probar las simulaciones y corregir bugs, 
            asi que se quedó e incluso va a formar parte del análisis a continuación.

            \item \textbf{Focus Strategy:} Fija desde el inicio a un reino enemigo y dirige sus ataques a este, turno tras turno hasta 
            que muera. Cuando su objetivo muere procede a buscar un nuevo objetivo, intentando que no sea un aliado, aunque si no queda 
            de otra no tendrá problemas en romper una alianza. Solo crea alianzas con reinos que no son el objetivo al que está atacando 
            en ese momento.

            \item \textbf{Multiple Strategy:} Recibe en su constructor una lista de estrategias y un nivel de importancia por cada una, 
            en cada turno elije la acción que elegiría una de esas estrategias dando más posibilidad a las estrategias más importantes 
            pero sin descartar ninguna. Esta estrategia es mas bien una familia de estrategias. Evidentemente esta estrategia depende
            de otras para funcionar y hay infinitas formas de meterla en una simulación (jugando con las estrategias y las importancias)
            pero al final su efectividad va a depender de las estrategias que la compongan. Por esto, y para ahorrar tiempo, elejimos 
            dejar esta estrategia fuera del análisis.

            \item \textbf{Random Strategy:} Como su nombre indica, juega de manera aleatoria. Esto incluye proponer alianzas a reinos 
            aleatorios y aceptar una solicitud de quien sea con un 50\% de probabilidad.

        \end{itemize}

            \subsection{Resultados para n=2}
            Se realizaron simulaciones haciendo todos los posibles emparejamientos entre pares de estrategias. Cada par se enfrentará 
            20 veces, 10 veces saliendo uno de primero y las otras 10 se invierte el orden. En este caso es importante analizar ambos 
            resultados ya que al ser solo 2 reinos, el primero que mueve tiene una ligera ventaja sobre el otro. Estos fueron los 
            resultados de las simulaciones:

            \begin{longtable}{|c|c|c|}
                \hline \textbf{Reino 1}  & \textbf{Reino 2}  & \textbf{Resultado} \\ 
                \endfirsthead
                \hline \textbf{Reino 1}  & \textbf{Reino 2}  & \textbf{Resultado} \\ 
                \endhead
                \hline      Allies       &   Always Attack   &   4 - 6   \\
                \hline   Always Attack   &      Allies       &   8 - 2   \\
                \hline      Allies       &    Attack Weak    &   3 - 7   \\
                \hline    Attack Weak    &      Allies       &   6 - 4   \\
                \hline      Allies       & Current Situation &   9 - 1   \\
                \hline Current Situation &      Allies       &   2 - 8   \\
                \hline      Allies       &      Defend       &  10 - 0   \\
                \hline      Defend       &      Allies       &   3 - 7   \\
                \hline      Allies       &    Do Nothing     &  10 - 0   \\
                \hline    Do Nothing     &      Allies       &  0 - 10   \\
                \hline      Allies       &       Focus       &   6 - 4   \\
                \hline       Focus       &      Allies       &   7 - 3   \\
                \hline      Allies       &      Random       &   8 - 2   \\
                \hline      Random       &      Allies       &   3 - 7   \\
                \hline   Always Attack   &    Attack Weak    &   7 - 3   \\
                \hline    Attack Weak    &   Always Attack   &   7 - 3   \\
                \hline   Always Attack   & Current Situation &   7 - 3   \\
                \hline Current Situation &   Always Attack   &   5 - 5   \\
                \hline   Always Attack   &      Defend       &   9 - 1   \\
                \hline      Defend       &   Always Attack   &   4 - 6   \\
                \hline   Always Attack   &    Do Nothing     &  10 - 0   \\
                \hline    Do Nothing     &   Always Attack   &  0 - 10   \\
                \hline   Always Attack   &       Focus       &   8 - 2   \\
                \hline       Focus       &   Always Attack   &   7 - 3   \\
                \hline   Always Attack   &      Random       &  10 - 0   \\
                \hline      Random       &   Always Attack   &   1 - 9   \\
                \hline    Attack Weak    & Current Situation &   8 - 2   \\
                \hline Current Situation &    Attack Weak    &   3 - 7   \\
                \hline    Attack Weak    &      Defend       &   9 - 1   \\
                \hline      Defend       &    Attack Weak    &   3 - 7   \\
                \hline    Attack Weak    &    Do Nothing     &  10 - 0   \\
                \hline    Do Nothing     &    Attack Weak    &  0 - 10   \\
                \hline    Attack Weak    &       Focus       &   5 - 5   \\
                \hline       Focus       &    Attack Weak    &   6 - 4   \\
                \hline    Attack Weak    &      Random       &  10 - 0   \\
                \hline      Random       &    Attack Weak    &   2 - 8   \\
                \hline Current Situation &      Defend       &   9 - 1   \\
                \hline      Defend       & Current Situation &   1 - 9   \\
                \hline Current Situation &    Do Nothing     &  10 - 0   \\
                \hline    Do Nothing     & Current Situation &  0 - 10   \\
                \hline Current Situation &       Focus       &   3 - 7   \\
                \hline       Focus       & Current Situation &   5 - 5   \\
                \hline Current Situation &      Random       &   8 - 2   \\
                \hline      Random       & Current Situation &   5 - 5   \\
                \hline      Defend       &    Do Nothing     &  10 - 0   \\
                \hline    Do Nothing     &      Defend       &  0 - 10   \\
                \hline      Defend       &       Focus       &   2 - 8   \\
                \hline       Focus       &      Defend       &   9 - 1   \\
                \hline      Defend       &      Random       &   5 - 5   \\
                \hline      Random       &      Defend       &   6 - 4   \\
                \hline    Do Nothing     &       Focus       &  0 - 10   \\
                \hline       Focus       &    Do Nothing     &  10 - 0   \\
                \hline    Do Nothing     &      Random       &  0 - 10   \\
                \hline      Random       &    Do Nothing     &  10 - 0   \\
                \hline       Focus       &      Random       &   9 - 1   \\
                \hline      Random       &       Focus       &   3 - 7   \\
                \hline
            \end{longtable}

            \vspace{5mm}

            \textbf{Resumen:} \vspace{5mm}

            \noindent En total se realizaron 560 simulaciones. Cada estrategia participa 
            en 140, de ellas, 70 saliendo de 1ro y 70 de 2do.

            \begin{longtable}{|c|c|c|c|c|}
                \hline \textbf{\#}  & \textbf{Estrategia}  & \textbf{Ganadas como 1ro} & \textbf{Ganadas como 2do} & \textbf{Total} \\ 
                \endfirsthead
                \hline \textbf{\#}  & \textbf{Estrategia}  & \textbf{Ganadas como 1ro} & \textbf{Ganadas como 2do} & \textbf{Total} \\ 
                \endhead
                \hline   1   &    Attack Weak    &        55        &        46        &  101  \\
                \hline   2   &   Always Attack   &        59        &        42        &  101  \\
                \hline   3   &       Focus       &        53        &        43        &  96   \\
                \hline   4   &      Allies       &        50        &        41        &  91   \\
                \hline   5   & Current Situation &        40        &        35        &  75   \\
                \hline   6   &      Random       &        30        &        20        &  50   \\
                \hline   7   &      Defend       &        28        &        18        &  46   \\
                \hline   8   &    Do Nothing     &        0         &        0         &   0   \\
                \hline
            \end{longtable}

            \subsection{Resultados para n=3}
            De forma similar al caso anterior se hicieron simulaciones con cada posible trío de 
            estrategias (20 simulaciones con cada uno) para probar su efectividad en todos los 
            escenarios posibles. La diferencia es que esta vez no consideramos todos los órdenes 
            posibles de salida principalmente para no tener un número excesivo de simulaciones, 
            para esto hicimos que antes de cada simulación se eligiera el orden de los turnos de 
            manera aleatoria, es decir, en las 20 simulaciones de un trio de estrategias estas 
            tienen un orden aleatorio, lo que nos permite hacer un análisis más general. Estos 
            fueron los resultados:

            \begin{longtable}{|c|c|c|c|}
                \hline \textbf{Reino 1}  & \textbf{Reino 2} & \textbf{Reino 3}  & \textbf{Resultado} \\ 
                \endfirsthead
                \hline \textbf{Reino 1}  & \textbf{Reino 2} & \textbf{Reino 3}  & \textbf{Resultado} \\
                \endhead
                \hline      Allies       &   Always Attack   &    Attack Weak    &  7 - 6 - 7  \\
                \hline      Allies       &   Always Attack   & Current Situation & 10 - 5 - 5  \\
                \hline      Allies       &   Always Attack   &      Defend       &  8 - 7 - 5  \\
                \hline      Allies       &   Always Attack   &    Do Nothing     & 6 - 14 - 0  \\
                \hline      Allies       &   Always Attack   &       Focus       &  6 - 7 - 7  \\
                \hline      Allies       &   Always Attack   &      Random       & 6 - 11 - 3  \\
                \hline      Allies       &    Attack Weak    & Current Situation & 11 - 6 - 3  \\
                \hline      Allies       &    Attack Weak    &      Defend       &  6 - 9 - 5  \\
                \hline      Allies       &    Attack Weak    &    Do Nothing     & 8 - 12 - 0  \\
                \hline      Allies       &    Attack Weak    &       Focus       & 5 - 11 - 4  \\
                \hline      Allies       &    Attack Weak    &      Random       &  7 - 8 - 5  \\
                \hline      Allies       & Current Situation &      Defend       &  7 - 6 - 7  \\
                \hline      Allies       & Current Situation &    Do Nothing     & 13 - 7 - 0  \\
                \hline      Allies       & Current Situation &       Focus       & 5 - 3 - 12  \\
                \hline      Allies       & Current Situation &      Random       & 11 - 6 - 3  \\
                \hline      Allies       &      Defend       &    Do Nothing     & 12 - 8 - 0  \\
                \hline      Allies       &      Defend       &       Focus       &  6 - 6 - 8  \\
                \hline      Allies       &      Defend       &      Random       & 7 - 11 - 2  \\
                \hline      Allies       &    Do Nothing     &       Focus       & 10 - 0 - 10 \\
                \hline      Allies       &    Do Nothing     &      Random       & 16 - 0 - 4  \\
                \hline      Allies       &       Focus       &      Random       &  6 - 9 - 5  \\
                \hline   Always Attack   &    Attack Weak    & Current Situation &  9 - 8 - 3  \\
                \hline   Always Attack   &    Attack Weak    &      Defend       &  9 - 8 - 3  \\
                \hline   Always Attack   &    Attack Weak    &    Do Nothing     & 15 - 5 - 0  \\
                \hline   Always Attack   &    Attack Weak    &       Focus       &  8 - 6 - 6  \\
                \hline   Always Attack   &    Attack Weak    &      Random       & 7 - 10 - 3  \\
                \hline   Always Attack   & Current Situation &      Defend       &  9 - 7 - 4  \\
                \hline   Always Attack   & Current Situation &    Do Nothing     & 14 - 6 - 0  \\
                \hline   Always Attack   & Current Situation &       Focus       &  9 - 4 - 7  \\
                \hline   Always Attack   & Current Situation &      Random       & 13 - 7 - 0  \\
                \hline   Always Attack   &      Defend       &    Do Nothing     & 13 - 7 - 0  \\
                \hline   Always Attack   &      Defend       &       Focus       &  5 - 8 - 7  \\
                \hline   Always Attack   &      Defend       &      Random       & 11 - 7 - 2  \\
                \hline   Always Attack   &    Do Nothing     &       Focus       & 9 - 0 - 11  \\
                \hline   Always Attack   &    Do Nothing     &      Random       & 13 - 0 - 7  \\
                \hline   Always Attack   &       Focus       &      Random       & 10 - 6 - 4  \\
                \hline    Attack Weak    & Current Situation &      Defend       & 10 - 7 - 3  \\
                \hline    Attack Weak    & Current Situation &    Do Nothing     & 17 - 3 - 0  \\
                \hline    Attack Weak    & Current Situation &       Focus       &  7 - 6 - 7  \\
                \hline    Attack Weak    & Current Situation &      Random       & 11 - 4 - 5  \\
                \hline    Attack Weak    &      Defend       &    Do Nothing     & 12 - 8 - 0  \\
                \hline    Attack Weak    &      Defend       &       Focus       & 10 - 4 - 6  \\
                \hline    Attack Weak    &      Defend       &      Random       & 15 - 1 - 4  \\
                \hline    Attack Weak    &    Do Nothing     &       Focus       & 7 - 0 - 13  \\
                \hline    Attack Weak    &    Do Nothing     &      Random       & 16 - 0 - 4  \\
                \hline    Attack Weak    &       Focus       &      Random       & 13 - 5 - 2  \\
                \hline Current Situation &      Defend       &    Do Nothing     & 12 - 8 - 0  \\
                \hline Current Situation &      Defend       &       Focus       &  6 - 5 - 9  \\
                \hline Current Situation &      Defend       &      Random       & 5 - 11 - 4  \\
                \hline Current Situation &    Do Nothing     &       Focus       & 7 - 0 - 13  \\
                \hline Current Situation &    Do Nothing     &      Random       & 12 - 0 - 8  \\
                \hline Current Situation &       Focus       &      Random       & 7 - 11 - 2  \\
                \hline      Defend       &    Do Nothing     &       Focus       & 6 - 0 - 14  \\
                \hline      Defend       &    Do Nothing     &      Random       & 12 - 0 - 8  \\
                \hline      Defend       &       Focus       &      Random       &  8 - 9 - 3  \\
                \hline    Do Nothing     &       Focus       &      Random       & 0 - 17 - 3  \\
                \hline
            \end{longtable}

            \newpage

            \textbf{Resumen:} \vspace{5mm}

            \noindent Se realizaron en total 1120 simulaciones, cada estrategia participa en 420.

            \begin{longtable}{|c|c|c|}
                \hline \textbf{\#}  & \textbf{Estrategia}  & \textbf{Victorias} \\ 
                \endfirsthead
                \hline \textbf{\#}  & \textbf{Estrategia}  & \textbf{Victorias} \\  
                \endhead
                \hline    1   &    Attack Weak    &    208    \\
                \hline    2   &   Always Attack   &    204    \\
                \hline    3   &       Focus       &    191    \\
                \hline    4   &      Allies       &    173    \\
                \hline    5   &      Defend       &    137    \\
                \hline    6   & Current Situation &    126    \\
                \hline    7   &      Random       &    81     \\
                \hline    8   &    Do Nothing     &     0     \\
                \hline
            \end{longtable}

            \subsection{Análisis de los resultados}
            A simple vista se aprecia que no hacer nada es una terrible estrategia, y la estrategia random 
            no es para nada viable. Luego podemos ver que las estrategias más ofensivas tienen un mejor 
            rendimiento para un número bajo de reinos, y tomar iniciativa primero suele dar una ligera 
            ventaja. Los reinos mas pasivos y las estrategias más elaboradas suelen tener mejor rendimiento a 
            medida que el número de reinos aumentan. Otro dato curioso es que la mayoria de victorias de las
            estrategias centradas en el aguante (dígase Current Situation y Defend) fueron por score, o sea, por
            límite de rondas, lo que indica que el límite de rondas que se defina puede afectar el porcentaje
            de victorias de estas 2 estrategias.

        \newpage

        \section{Conclusiones}
        Nuestro trabajo consistió en averiguar el rendimiendo de diversas estrategias a seguir en conflictos 
        medievales. Para esto diseñamos un modelo simple y escalable con el propósito de simular gran variedad 
        de escenarios para poner en práctica diferentes acciones a escoger por sus respectivos reyes, representados 
        como agentes que emplean conocimiento en la búsqueda del mejor final de turno de acuerdo a su propia 
        estrategia. Para el mejor estudio de los resultados de las simulaciones, se utilizo LLM en el procesamiento 
        de los logs de la simulación para hacerlos mas legibles mediante una historia creada a partir de estos. Luego, 
        como resultados de las simulaciones hechas concluimos que es más recomendable ser más agresivo cuando el número 
        de reinos es bajo, una actitud más prudente parecida a la de estrategia Allies si el número es elevado, o una 
        actitud de más aguante si el límite de rondas es bajo. Aun así, el equipo considera que se deben probar mayor 
        número de escenarios con diferentes estrategias a las probadas.
        
\end{document}